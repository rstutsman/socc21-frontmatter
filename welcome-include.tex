\section*{Welcome from the Chairs}

\setlength{\parindent}{0em}
\setlength{\parskip}{1em}
\newcommand{\rf}[1]{\textcolor{red}{(#1 -- Rodrigo)}}

Dear colleagues,

Welcome to SoCC 2021!
SoCC 2021 is the twelfth annual ACM Symposium on Cloud Computing, the premier conference on cloud computing. It brings together researchers, developers, end-users, and practitioners interested in wide-ranging aspects of cloud computing, and it is the only conference co-sponsored by the ACM Special Interest Groups on Management of Data (SIGMOD) and on Operating Systems (SIGOPS). 

This year, we are holding the first hybrid SoCC conference:  three days virtual + in-person (at the Renaissance Hotel in Seattle, Washington with limited attendees, November 1\textsuperscript{st} through 3\textsuperscript{rd}) and a fourth day, November 4\textsuperscript{th}, an all virtual event. We are using Whova and Teams for online participation; the in-person experience will be live streamed and remote registered audience members will be able to engage on video and chat. We are also live streaming on a public channel (limiting interactions to monitored chat) to maximize accessibility. All content (unless explicitly forbidden by the presenter) will be recorded and made available for on-demand viewing. To democratize access to SoCC, we have worked hard to make online and in-person pricing as low as possible (substantially lower than most comparable conferences) this is thanks to generous donations from our sponsors and a strong team of negotiators at ACM. 

We learned from last year's all virtual SoCC that this format brings new people to the conference who have never attended SoCC and would not have otherwise attended. We anticipate that going forward conferences will always have an online component with much wider participation. We are using this year to learn as much as we can about what works and what doesn't work in the hybrid format.  

SoCC is committed to bringing the diverse community of researchers, students, end-users, and practitioners together and making the conference an inclusive experience for everyone. To this end, SoCC started having a dedicated D\&I chair position in 2019, and we have continued this tradition this year as well. There were a great set of D\&I initiatives for the in-person conference in 2019 and for the online version of SoCC in 2020. With 2021 being the first hybrid event for SoCC, we have been working on creating a D\&I plan to suit the online and in-person attendees that includes student scholarships, a mentorship program, video captioning, and the involvement of the D\&I chair in influencing the conference decisions from a D\&I perspective. 

This year we selected 46 papers: 40 research papers, 4 vision papers, and  2 industry papers. SoCC 2021 continues the conference's tradition of incorporating broad (in scope) research that encompasses diverse data management and systems topics. Indeed, the selected papers cover topics including serverless platforms, cloud management (by platforms and users, and in terms of emerging metrics such as energy usage), running cloud paradigms at the edge, tracing, fault tolerance and testing, DB and query optimization, security monitoring and preservation, microservice management, and efficient/robust machine-learning training and inference.

The conference also includes four keynotes by leaders in the field: Aditya Akella (UT Austin), Peter Bailis (Sisu Data), Ranjita Bhagwan (Microsoft Research, India), and Krysta Svore (Microsoft). Our keynote selection process focused not only on topical fit but also on balancing representation in terms of gender, background, and vantage point (academia versus industry). The conference concludes with a panel.

There were 145 papers submitted to SoCC this year. Authors were invited to submit a paper in one of the following categories: (a) Research Papers (up to 12 pages) that describe original research, where novelty is a primary consideration, (b) Industry/System Papers (up to 12 pages) that describe important industrial advances and application achievements, and (c) Vision Papers (up to 6 pages) that describe speculative but well-reasoned and thought-provoking ideas, where insight is a primary consideration. Of the 145 submissions, 18 were vision papers, 13 were industry papers, and the remaining 115 were research papers.

Submissions were judged on their originality, technical merit, relevance, value to the community, and likelihood of producing interesting discussions at the symposium.  A program committee of 62 leading scholars from the computer systems and data management communities conducted a two-stage review process with extensive online discussion. Each submission received between three and six PC reviews, and PC members actively discussed responses provided by authors during the ``Author Response Period'' (after reviews were completed but before paper decisions were made). The PC made every effort to provide objective and useful feedback. The extensive online discussion allowed us to resolve many papers ahead of time, only leaving contentious papers for the online PC meeting. The final decisions were made at the online PC meeting to accept 40 research, 2 industry/system, and 4 vision papers. PC members did not participate in deliberations of any paper with which they had a conflict of interest. To ensure the highest quality of papers in the proceedings, all accepted papers were shepherded by a PC member.

We thank all of the authors who submitted their best work to SoCC. It is the creative and exciting research represented by these submissions that enabled us to build a strong program of 46 papers (compared to 35 papers selected last year) that uphold the high standard of SoCC, while actively growing the community.

We are grateful to our Program Committee for the many hours they spent reviewing and discussing papers, for putting so much effort into providing detailed and high-quality reviews, for their active participation in the discussions, and for helping select the best papers. We also thank the PC members who shepherded a paper to ensure that the final program embodies the strongest possible version of each accepted paper.

We thank our amazing organizing committee for their patience, creativity, and passion in making every detail come together: Rodrigo Fonseca (for all local arrangements and for helping on so much more), Chris Douglas (for thoughtful budgeting), Neeraja Yadwadkar (for all D\&I aspects),  Supreeth Shastri (for an up-to-date web site) , Renata Borovica-Gajic (for running registrations), Pınar Tözün (for lots of publicity), Manos Athanassoulis (for ensuring strong sponsor support), Peter Alvaro (for organizing mentoring and job seeking), and Ryan Stutsman (for running the publication process). We also thank you for your invaluable participation in all of our meetings over the past many months.  Additionally, we would like to express our gratitude to the fantastic Steering committee -- Carlo Curino, Rodrigo Fonseca, Shahram Ghandeharizadeh, Phillip B.\ Gibbons and Hakim Weatherspoon -- for their advice and active support. 

We are very thankful for the generous support from this year's corporate sponsors (Microsoft, Salesforce, Exotanium, Amazon, MongoDB, TileDB, IBM, Oracle, CISCO, and Google), who decided to sponsor the conference despite the uncertainties that 2021 still brings. 

Eddie Kohler's HotCRP conference management system was crucial for submissions, revisions, and in the preparation of these proceedings. Finally, our deepest thanks to all of the attendees and participants in this year's conference.   We hope you will enjoy the SoCC 2021 program!

\begin{flushright}
Carlo Curino\\
\emph{2021 SoCC General Chair}\\\vspace{0.5em}
Georgia Koutrika \& Ravi Netravali \\
\emph{2021 SoCC Program Chairs}
\end{flushright}


%logistics
